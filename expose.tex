\documentclass{beamer}
\usepackage[utf8]{inputenc}
\usepackage[T1]{fontenc}
\usepackage[french]{babel}
\usepackage{amsmath,amsfonts,amssymb}
\usepackage{graphicx}

\usetheme{Warsaw}

\author{Aurèle Barrière \& Nathan Thomasset}
\title{Cryptographie}
\date{10 mars 2016}


\begin{document}

\begin{frame}
\maketitle
\end{frame}

\begin{frame}{Mise en situation}
  \end{frame}

\begin{frame}{Intérêt de la cryptographie}
  \begin{itemize}
  \item Cartes bleues
    
  \item Mail
    
  \item Transactions bancaires
  
  \item Chiffrement des données sensibles (militaires ou privées)
\end{itemize}
\end{frame}

\begin{frame}{Un codage ultime?}
  Seul quelqu'un qui connaitraît la clé pourrait décoder : est-ce réellement possible ?
  \end{frame}

\begin{frame}{Exemple : chiffrement de César}
  Décalage constant. 
$$A\rightarrow B, B\rightarrow C, \text{...}$$ 
$$A\rightarrow C, B\rightarrow D, \text{...}$$

  \begin{figure}
    \centering
  \includegraphics[scale = 0.25]{cesar.png}
\end{figure}
\end{frame}

\begin{frame}{Veni Vidi Decrypti}
  Vingt-six possibilités pour le décalage utilisé : il est possible de toutes les tester.
  \end{frame}

\begin{frame}{Énumération des clés}
  \begin{itemize}
  
  \item Il est possible d'énumérer toutes les clés.
  
  \item Dans la majorité des algorithmes employés, l'ensemble des clés est fini.
  
  \item Même si ce n'est pas le cas, la mémoire allouée au stockage de la clé est limitée : l'ensemble des clés utilisables est fini.
  
  \end{itemize}
  \end{frame}

\begin{frame}{Complexité}
  L'objectif n'est pas de créer un chiffrement incassable, mais un chiffrement qui soit trop coûteux à casser.
  
  \begin{figure}
  \centering
  \includegraphics[scale = 0.35]{xkcdpassword_strength.png}
  \end{figure}
  \end{frame}

\begin{frame}{D'autres exemples}
  Hill

  Vigenere
  \end{frame}

\begin{frame}{Analyse fréquentielle}
  Chiffres pour matrices de Hill

  Fréquences français
  \end{frame}


\begin{frame}{Cryptographie asymétrique}
  Clé publique, clé privée

  Mise en situation
\end{frame}

\begin{frame}{RSA}
  Schéma
  \end{frame}

\begin{frame}{Limites}
  \begin{figure}
    \centering
    \includegraphics[scale = 0.5]{xkcdsecurity.png}
  \end{figure}
\end{frame}

\begin{frame}{Ressources et idées}
  GPG mail
  
  sources des images
  \end{frame}





\end{document}
